\documentclass{article}
\usepackage{multirow,tabularx}
\usepackage{amsmath}
\usepackage{amssymb}
\pagestyle{empty}
\setlength\parindent{0pt}
\usepackage[paperwidth=5in, 
    paperheight=3.5in,
    left=0.10in,
    right=0.10in,
    top=0.10in,
    bottom=0.10in]{geometry}
\renewcommand{\arraystretch}{1.3}
\begin{document}

%73 from KE8QZC of Fairmont, WV
\vspace*{0.01in}
\begin{center}
\begin{tabular}{|l|l|l|l|l|l|l|}
\hline
\multicolumn{2}{|c|}{QSO with} & D / M / Y & UTC & Freq & RST & Mode \\
\hline 
\multicolumn{2}{|c|}{\phantom{.....}} & & & & & \\
\hline
\multicolumn{2}{|c|}{Rig} & \multicolumn{2}{c|}{Antenna} & Watts & QSL & Commit* \\
\hline 
\multicolumn{2}{|c|}{\phantom{.....}} & \multicolumn{2}{c|}{\phantom{.}} &   & PSE: $\square$ TNX: $\square$ &  \\
\hline
\multicolumn{5}{|c|}{QTH during contact} & \multicolumn{2}{c|}{Grid square} \\ 
\hline 
\multicolumn{5}{|c|}{\phantom{.}} & \multicolumn{2}{c|}{\phantom{.}} \\
\hline
\end{tabular} 
\end{center}
\vfill
{\small *the front of this QSL card is unique to your callsign -- see the webpage \\ \texttt{https://github.com/tomcuchta/hypergeometricqsl} for an explanation!}
\end{document}